\documentclass[a4paper,12pt]{article}


\textwidth=480pt
\textheight=680pt
\oddsidemargin=-10pt
\evensidemargin=-10pt
\topmargin=-0.8in


\usepackage{epsfig}
\usepackage{graphicx}
\usepackage{graphics}

\usepackage{bm}
\usepackage{amsmath}
\usepackage{amssymb}
\usepackage{amsfonts}
%\usepackage[top=0.4in,bottom=1.6in,right=1.in,left=1.in]{geometry}

\begin{document}
\title{\bf Energy convergence of the H$_2^+$ ion: A FE-DVR Approach in prolate
spheroidal coordinates}
\author{Xiaoxu Guan}

%\maketitle

\noindent


\begin{table}[ht]
\caption{\label{tab:energy+} Convergence of two-electron Coloumb potential
$1/r_{12}$ with respect to $\ell_{\rm max}$ in the Neumann expansion in
prolate spheroidal coordinates. Internuclear separation $R$ is $1.4$. All the
quanties are given in atomic units.}
\begin{tabular}{lcllllc}
\hline\hline
Electron $1$&   & $(0.2,-0.3,0.0)$& &  $(1.0,-2.0,-1.0)$& &  $(10,-2,-50)$ \\
Electron $2$&   & $(0.5,0.1,-1.0)$&  & $(-1.0,3.0,1.0)$& &$(5,1,-3)$   \\ \hline
$\ell_{\rm max}=10$ & & $0.8944912425297$& & $0.1759041023385$& &
                       $0.02111472177686$      \\
$\ell_{\rm max}=20$ &  & $0.8944272224030$& &  $0.1739625665290$& & 
		       $0.02111472177657$   \\
$\ell_{\rm max}=30$&   & $0.8944271910024$& & $0.1740764143734$&&
                       $0.02111472177657$  \\
$\ell_{\rm max}=40$&   & $0.8944271909999$& & $0.1740778613865$& &
                       $0.02111472177657$  \\
$\ell_{\rm max}=50$&  & $0.8944271909999$&& $0.1740776557020$& &
                       $0.02111472177657$\\   \hline
Exact $1/r_{12}$&   & $0.8944271909999$& &  $0.1740776559557$&&
$0.02111472177657$
  \\ \hline\hline
\end{tabular}
\end{table}
The von Neumann expansion of $1/r_{12}$ in prolate spheroidal coordinates is
given by
\begin{equation}
\frac{1}{r_{12}}=
\frac{2}{R}\sum_{\ell=0}^{\infty}\sum_{m=-\ell}^{\ell}(-1)^{|m|}(2\ell+1)
\bigg(\frac{(\ell-|m|)!}{(\ell+|m|)!} \bigg)^2
P_{\ell}^{|m|}(\xi_<)Q_{\ell}^{|m|}(\xi_>)P_{\ell}^{|m|}(\eta_1)P_{\ell}^{|m|}
(\eta_2)
e^{im(\varphi_1-\varphi_2)},
\end{equation}
where $\xi_{<}=\min(\xi_1,\xi_2)$ and $\xi_{>}=\max(\xi_1,\xi_2)$.


\end{document}
