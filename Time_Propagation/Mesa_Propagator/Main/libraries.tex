\documentclass[onecolumn,letterpaper]{report}
\title{Codes and Libraries Developed by BIS and Collaborators }
\author{Barry Schneider}
\date{\today}

\usepackage{graphics}
\usepackage{epsfig}
\usepackage{amssymb}    % To get extra math characters
\usepackage{amsbsy}     % To get \boldsymbol
\usepackage{amstext}    % To get \text
\usepackage{psfrag} 
\usepackage{color}  
\newcommand{\ie}{\emph{i.e.} } 
\newcommand{\eg}{\emph{e.g.} } 
\newcommand{\ua}{\uparrow} 
\newcommand{\da}{\downarrow} 

\begin{document}
\maketitle
\section{Preamble}
Back in the middle 1980's, Byron Lengsfield, Richard Martin, Barry Schneider and a
number of other researchers at LANL, LLNL and other laboratories began to develop a 
new electronic structure code which they called MESA.  The object was to make this 
code clear, very portable, modular and to have it well documented.  To accomplish that, 
the developers provided a library which contained all of the commonly used mathematical 
routines, a library which performed commonly used operations on strings and a user 
friendly input/ouput system.  These libraries contained a set of subroutines which would 
allow users to perform;
\begin{itemize}
\item Mathematical manipulations on functions, vectors and 
matrices
\item User friendly random I/O
\item Character manipulations on both input and output variables 
\item Other operations which are repeatedly used across a broad
array of scientific programs
\end{itemize}
The library grew over the years to incorporate as many 
of the standard Fortran libraries as are openly available so that a user in 
moving code from one platform to another would not be subject to any availabilty issues 
which would prevent their code from running on that platform.  Thus, the current 
version includes all of the BLAS, LAPACK, EISPACK, LINPACK, CLAMS and
SLATEC libraries as well as many other specialized routines.  In addition, it includes a 
number of subroutines required for machine dependent manipulations.  Many of these are now
obsolete or were developed for platforms which are no longer being maintained.  In other 
cases, the differences between platforms have vanished and there are generic routines 
which transcend specific architectures.  For the BLAS/LAPACK/LINPACK routines, there is 
an option in the Makefile to use machine specific libraries, where they are available.
The best advice to a novice is to look at the routines in the sublibraries.  They are 
generally well commented internally or refer to outside sources.  The intent is for the 
user to compile the sublibraries using the Makefiles provided and to then use the total 
mesalib.a in whatever applications are being developed.  New routines may be added to 
the library but they should follow the conventions established by the developers so 
that new users can transparently and easily use the routines.  
Note, there is no attempt to provide detailed documentation of all of the subroutines 
independently of actually looking at the source code.
\par
Each of the subroutine libraries described below has a Makefile associated with it.  The 
compiled routines are all placed in one global library called mesalib.a.  On some
platforms, the order in which these libraries are made is important as externals may not
be resolved under certain circumstances.  I have never been able to figure out the
rhyme or reason for this but the following order should work;
\begin{itemize}
\item Character \_ Manipulation \_ Subroutines
\item  Common \_ Los \_ Alamos \_ Mathematical \_ Subroutines
\item General \_ Utility \_ Subroutines
\item IOsys \_ Subroutines
\item Mathematical \_ Subroutines 
\item Machine \_ Dependent \_ Subroutines
\end{itemize}
\section{Machine \_ Dependent \_ Subroutines}
This consists of a small number of subroutines which originally depended on 
specific machine characteristics.  To a large extent, this is no longer true and
most if not all of the routines will run on all architectures.  The subdirectory
labelled opteron may be used for most machines.  There is some editing of a few
of these subroutines for specific machines that will add headers to your output
which tell you which machine you are running on and your user name as well as some
other things.  The routines to look at are versn.f, usrnam.f, lxopen.f and getmem.f.
The changes that are required are easy to see.
\section{Character\_ Manipulation\_ Subroutines}
This is a set of routines to manipulate character variables.  It is used by many
of the IOsys routines and is important in freeing the user from worry about
how to structure their input decks.  Since input is keyword driven 
and variables are entered in ``english'', it is necessary to decode the strings and
get the numerical values entered into the correct places.  As an example consider
the following part of an input file used in the propagation code,
\newline
\$prop\_ basis
\newline
                 type-calculation=imaginary-time
\newline
 number-of-space-variables=1 coordinate-system=cartesian
\newline
 space-variable-1=x use-atomic-units kinetic-energy-type=packed
\newline
 xplot get-eigenpairs diagonal-modification=none
\newline
 plot\_ step=1 xprint=all-details eigenvalue-convergence=1.d-10
\newline
 propagation-order=4 xpropagation-method=arnoldi
\newline
\$end
\newline
What would happen in the read is that the input file would be read until it encountered the
keyword {\$prop\_ basis}.  Then it would read everything up to the \$end and enter the values
into the appropriate variables.  In most cases, there are default values if a variable
is omitted.  This keyword sequence may be placed anywhere in the input file.   Note, that an x
in front of a keyword is just a convenient way to not set it without erasing it from the file.
\section{IOsys\_ Subroutines}
This contains the subroutines to perform user friendly I/O operations on files.
The rules for opening, reading, writing and making inquiries to IOsys files are described 
quite well in the main routine iosys.f.  The main point for this discussion
is that all of the IO done using IOsys is random IO but the user is freed from worrying about
any of the issues that would be typical of such IO.  For example, since files are written to units
with names which can be quite descriptive of what the file contains, (i.e 'Hamiltonian Matrix') the
user does not have to know where it appears in the unit.  It is even possible to use an option
which does not require the user to know how many words are on the file, only the type, such as
real, integer or character.
\section{Common\_ Los\_ Alamos\_ Mathematical\_ Subroutines}
This contains source code for the entire set of mathematical subroutines called
CLAMS.  The source code for this library was taken from the Los Alamos library.  It is
open source and non-propritary.  Note that there are typically machine language routines
for many of these routines and there is an option to use those if they are available.
\section{Mathematical\_ Subroutines}
Many additional mathematical routines which are used by applications developers.  These have been
developed in the course writing application specific codes.  One simple example is a set of
routines to multiply matrices, matrices times their transposes, matrix transposes times matrix
tranposes, matrix addition, quadrature routines, etc.  The names of these routines were intended
to be descriptive and easy for the user to recall.
\section{General\_ Utility\_ Subroutines}
Routines to print matrices and to do lots of operations needed in applications.
\section{Environment}
I find it useful to set up environment variables which define various directories that are
used, the name and location of the compilers and/or preprocessors and the paths.  My shell of
choice is the bash shell but that is a personal one.  What follows can be adapted to other 
shells with minor changes.
\newline
Here is my .bash\_ profile file which is exectuted when I log on to the workstation.
\newline
\begin{verbatim}
# .bash_profile

# Get the aliases and functions
if [ -f ~/.bashrc ]; then
	. ~/.bashrc
fi

# User specific environment and startup programs
# Export the Mesa Directories

export MAKEFLAGS=
export MESA_HOME=/home/mesa/mesa
export MESA_LIB=$MESA_HOME/library
export MESA_BIN=$MESA_HOME/bin
export CPPEXT=.cpp
export FFLAGS='-c'
#export FC='/opt/intel/fce/9.0/bin/ifort'
export FC=ifort
export MDLIB='/opt/intel/mkl721/lib/em64t/libmkl_em64t.a \
              /opt/intel/mkl721/lib/em64t/libmkl_lapack.a \
              /opt/intel/mkl721/lib/em64t/libguide.a \
              /opt/intel/mkl721/lib/em64t/libmkl_solver.a'
export CPP=/usr/bin/cpp
export CPPFLAGS='-DMACHINEBLAS -DMACHINELAPACK'
export TMP=/tmp
TLMHOST=@feynman.mps.nsf.gov
export TLMHOST
PATH=$PATH:$HOME/bin:$MESA_HOME:$MESA_RUN:$MESA_BIN:$TMPDIR:$TEC90HOME/bin
PATH=$PATH:$MESA_TMP:$MESA_LIB:$CPP:$HOME/LaTeX:/sbin:/usr/local/bin
BASH_ENV=$HOME/.bashrc
export BASH_ENV PATH
unset USERNAME
\end{verbatim}
All of these are compiled into the library mesalib.a


\section{DVRLIB}
This library constructs DVR functions, first and second derivatives for a number of DVR
representations.  In addition, the library also provides the necessary matrix elements
to construct the kinetic energy matrix and many one dimensional potential energy matrix elements.
These are also available for the FEDVR based on Legendre functions.  We now consider a special 
class of DVRs, where the underlying spectral basis functions are defined in terms of a 
classical orthogonal polynomial such that
\begin{equation}
\phi_n(x) = \sqrt{w(x)} \, p_{n-1}(x),
\label{basisfcts}
\end{equation}
where $w(x)$ is a non-negative weight function, and
$\{p_n,n=0,1,\ldots \}$ are polynomials of degree $n$, orthogonal to
each other over a range $[a,b]$ with respect to $w(x)$, \ie
\begin{equation}
\int_a^b dx \, w(x) p_m(x) p_n(x) = \delta_{mn}.
\end{equation}
Depending on the orthogonal polynomial in question the interval can be
either finite, semi-infinite or infinite. Including the weight function
directly in the basis functions slightly modifies the definition of
the DVR functions:
\begin{equation}
\chi_{\alpha} (x) = \sum_{n=1}^N  \phi_n(x) \phi^*_n (x_\alpha)) 
\sqrt { \frac{ w(x_\alpha)}{ w_\alpha }  }
\label{basis}
\end{equation}
as is easily demonstrated.
The DVR functions are therefore the product of a polynomial of order
$N-1$ by the square root of the weight function. 
The special property of the orthogonal polynomials is that the
associated quadrature is of the Gauss-Jacobi type. Hence for a
quadrature specified on $N$ points the integration rule
is {\emph{exact}} when the integrand is a polynomial
of degree $2N-1$ or less~\cite{koonin1990,Gabor1939}. 
We will now determine the DVR grid points and the quadrature weights. 

The orthogonal polynomials are known to satisfy the
Christoffel-Darboux sum formula~\cite{Gabor1939,Abramowitz1970}
\begin{equation}
\sum_{m=0}^{n} \, p_m(x) p_m(y)=\frac{k_{n}}{k_{n+1}} \,
\frac{p_{n+1}(x)p_{n}(y) 
- p_{n}(x) p_{n+1}(y)}{x-y},
\label{CD}
\end{equation}
where $k_n$ is the coefficient of $x^n$ in $p_n(x)$. An equivalent
formula holds for the basis functions~(\ref{basisfcts}) 
\begin{equation}
\sum_{m=1}^{N} \, \phi_m(x) \phi_m(y)=\frac{k_{N-1}}{k_{N}} \,
\frac{\phi_{N+1}(x)\phi_{N}(y) 
- \phi_{N}(x) \phi_{N+1}(y)}{x-y},
\label{CD2}
\end{equation}
as is seen by
multiplying both sides of~(\ref{CD}) by $\sqrt{w(x)w(y)}$. The
condition of discrete completeness of
the basis functions $\phi_n$ is then fulfilled for a given family of
orthogonal polynomials if  
\begin{equation}
\phi_{N+1}=0  \ \Longrightarrow \ p_N(x_{\alpha})=0.
\end{equation}
This defines the mesh points. To determine the
quadrature weights we follow the discussion in  
Szeg\"o~\cite{Gabor1939}: Setting $y=x_{\alpha}$ in (\ref{CD2}),
and integrating over the interval. We evaluate the left hand side
first:
\begin{eqnarray}
\int_{a}^{b}dx \, \sum_{m=1}^{N}\phi_m(x)\phi_m(x_{\alpha}) & = & 
\sum_{n=0}^{N-1} \left( \int_{a}^{b} dx \, w(x) p_n(x) p_0(x) \right)
\, \frac{p_n(x_{\alpha})}{p_0(x)} \nonumber \\
\mbox{} & =& \sum_{n=0}^{N-1} \delta_{n0} \,
\frac{p_n(x_{\alpha})}{p_0(x)} = 1,
\end{eqnarray}
using the fact that $p_o(x)$ is constant and non-vanishing.
On the right hand side the second term vanishes since
$\phi_{N+1}(x_{\alpha})=0$, and we are left with  
\begin{eqnarray}
1 &=& \frac{k_{N-1}}{k_{N}}\int_{a}^{b}dx \,
\frac{\phi_{N+1}(x)\phi_{N}(x_{\alpha})}{x-x_{\alpha}} \nonumber \\
\mbox{} &=&\frac{k_{N-1}}{k_{N}}\int_{a}^{b}dx \, w(x)
\frac{p_{N}(x) p_{N-1}(x_{\alpha})}{x-x_{\alpha}} =
\frac{k_{N-1}}{k_{N}} \sum_{\beta} \, w_{\beta} \frac{p_{N}(x_{\beta})
p_{N-1}(x_{\alpha})}{x_{\beta}-x_{\alpha}} \nonumber \\
\mbox{} &=& \frac{k_{N-1}}{k_{N}} p_N'(x_{\alpha}) p_{N-1}(x_{\alpha})
w_{\alpha}. 
\end{eqnarray}
The weights are then given by
\begin{equation}
w_{\alpha} = \frac{k_{N}}{k_{N-1}} \frac{1}{p_{N-1}(x_{\alpha})
p_{N}'(x_{\alpha})}.
\end{equation}
Going back to the expansion of the coordinate eigenfunctions
(\ref{basis}) we find from the Christoffel-Darboux formula for the
basis functions~(\ref{CD2}) that the DVR functions can be expressed as 
\begin{equation}
\chi_{\alpha}(x) = \sqrt{\frac{w_{\alpha}}{w(x_{\alpha})}} \,
\frac{k_{N-1}}{k_{N}} \,
\frac{\phi_{N+1}(x)\phi_{N}(x_{\alpha})}{x-x_{\alpha}}  
=\frac{p_{N}(x)}{p_{N}'(x_{\alpha})(x-x_{\alpha})}
\sqrt{\frac{w(x)}{w_{\alpha}}}.  
\label{chi_orthopol} 
\end{equation} 

We note
that the orthogonal polynomial of order $N$ can expressed as
\begin{equation}
p_N(x)=k_N\prod_{\alpha=1}^{N}(x-x_{\alpha}),
\end{equation}
where $p_N(x_{\alpha})=0$. The derivative of $p_N(x)$ is easily
calculated 
\begin{equation}
\phi_N'(x)=\sum_{{\alpha}=1}^{N}\frac{p_N(x)}{x-x_{\alpha}},
\end{equation}
and specifically on the mesh points it takes the values
\begin{equation}
\phi_N'(x_{\alpha})=k_N{\prod_{\beta=1}^{N}}{}' \,
(x_{\alpha}-x_{\beta}), 
\end{equation}
where the prime denotes the exclusion of the point
$x_{\beta}=x_{\alpha}$ in the  
product. The expression for the DVR functions can therefore be reduced
to the simple form
\begin{equation}
\chi_{\alpha}(x)=v_{\alpha}(x)\sqrt{\frac{w(x)}{w_{\alpha}}},
\label{DVRfunc}
\end{equation}
where $v_{\alpha}(x)$ are the 
Lagrange interpolating functions defined at the associated quadrature
points 
\begin{equation}
v_{\alpha}(x)={\prod_{\beta=1}^{N}}{}' \,
\frac{x-x_{\beta}}{x_{\alpha}-x_{\beta}}. 
\end{equation}
Matrix elements of the derivative operator are easily found
as $v_{\alpha}(x)$ and its derivatives can be tabulated. We remark
that an alternative, yet less straightforward way to evaluate the
kinetic energy matrix elements, is to use the known properties of
the orthogonal polynomials and their derivatives. For each particular
orthogonal polynomial DVR analytical formulas for $T_{\alpha\beta}$
can then be derived~\cite{Baye1986}. 

\begin{thebibliography}{1}

\bibitem{Baye1986}
D. Baye and P.-H. Heenen, J. Phys. A: Math. Gen. {\bf 19},  2041  (1986).

\bibitem{Light2000}
J.~C. Light and T. Carrington, Jr., Adv. Chem. Phys. {\bf 114},  263  (2000).

\bibitem{koonin1990}
S.~E. Koonin and D.~C. Meredith, {\em Computational Physics} (Addison-Wesley,
  Reading, Masachusetts, 1990).

\bibitem{Lemoine1994}
D. Lemoine, J. Chem. Phys. {\bf 101},  1  (1994).

\bibitem{Arfken1995}
G.~B. Arfken and H.~J. Weber, {\em Mathematical Methods for Physicists}, 4th
  ed. (Academic Press, San Diego, 1995).

\bibitem{Gabor1939}
G. Szeg\"o, {\em Orthogonal Polynomials}, Vol.~XXIII of {\em American
  Mathematical Society Colloqium Publications} (American Mathematical Society,
  New York, 1939).

\bibitem{Abramowitz1970}
M. Abramowitz and I. Stegun, {\em Handbook of Mathematical Functions} (Dover,
  New York, 1970).

\end{thebibliography}
\bibliographystyle{prsty}

\end{document}
