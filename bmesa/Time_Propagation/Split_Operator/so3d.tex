 \documentclass{article}
 \usepackage{graphicx}
 \setkeys{Gin}{width=\linewidth}
 \title{Main Routine to Drive Propagation}
 \author{Barry I. Schneider}
 \date{}
 \def \<{\langle}
 \def \>{\rangle}
 \begin{document}
 \maketitle
\begin{verbatim}
*deck so3d.f 
***begin prologue     so3d
***date written       960718   (yymmdd)
***revision date               (yymmdd)
***keywords           time, dvr, split-operator
***                   
***author             schneider, b. i.(nsf)
***source             so3d
***purpose            time dependent schroedinger equation 
***                   using split operator with finite difference or 
***                   dvr space representation.
***references       
***routines called    iosys, util and mdutil
***end prologue       so3d
      subroutine so3d(pham,edge,key,typke,typpot,spac,coord,
     1                card,units,vtyp,scale,omega,width,
     2                shift,gamma,dim,nphy,ntreg,prn,proj,pnch)
      implicit integer (a-z)
      integer*8 pham
      real*8 edge, delt
      real*8 ham1, ham2, ham3, y, vt
      real*8 cnverg, thresh, energy
      real*8 scale, omega, width, shift, gamma, t0
      real*4 secnds, delta(20), time(20)
      character*(*) typke, key, typpot, card, coord, units, vtyp
      character*80 title, i0stat
      character*2 itoc
      logical spac, prn, prdvd, proj, pnch
      dimension edge(ntreg+1), pham(dim)
      dimension h0(4), eigv0(4), eig0(4)
      dimension q0(4), q(4), qwt(4)
      dimension pq(4), dpq(4), nphy(4), prn(*), key(4), typpot(4)
      dimension vtyp(2)
      common/io/inp, iout     
      pointer(ph1,ham1(1))
      pointer(ph2,ham2(1))
      pointer(ph3,ham3(1))
      pointer (py,y(1)), (py,iy(1))
\end{verbatim}
     The equation we are solving here is;
\begin{equation}
  \big ( i \frac{d}{dt} - H \big ) \mid \Xi (\bf{r},t) \rangle \big )
              = H \mid \Psi_{0} (\bf{r},t_{0} ) \rangle
\end{equation}
where we write,
\begin{equation}
   \mid \Psi( \bf{r},t ) \rangle = \mid \Psi_{0}( \bf{r},t_{0}) \rangle
                                   + \mid \Xi (\bf{r},t) \rangle
\end{equation}
 and $t_{0}$ is the initial time. For a time independent Hamiltonian or
 for small time steps, we have
\begin{eqnarray}
   \mid \Psi(\bf{r},t) \rangle &=& \mid \Psi_{0} (\bf{r},t_{0} ) \rangle
                               + \mid  \Xi(\bf{r},t) \rangle \\ \nonumber
                               &=& 
   \mid \Psi_{0} (\bf{r},t_{0} ) \rangle 
                       +  
   \big ( e^{iH(t-t_{0})} - 1 \big ) H \mid \Psi_{0} (\bf{r},t_{0} ) \rangle
\end{eqnarray}

     set the local pointers to their global value

\begin{verbatim}
      do 10 i=1,dim
         call setso(pham(i),h0(i),eigv0(i),eig0(i),q0(i),
     1              q(i),qwt(i),pq(i),dpq(i),nphy(i),prn,typke,spac)
 10   continue
      call vtim(vt,edge(1),key,typpot,reuse)
      ph1=pham(1)
      if(dim.gt.1) then
         ph2=pham(2)
      endif
      if(dim.gt.2) then
         ph3=pham(3)
      endif
\end{verbatim}
      dimensions in space
\begin{verbatim}
      n3d=1
      do 20 i=1,dim
         n3d=n3d*nphy(i)
 20   continue
      psi0=1 
      psi=psi0+2*n3d
      v=psi+2*n3d
      driver=v+n3d
      vexp=driver+2*n3d
      scr=vexp+2*n3d
      need=wpadti(scr+2*n3d)
      call memory(need,py,wdpsi,'psi',0)
      reuse=.false.
\end{verbatim}
     set up a scratch file to hold the wavefunction for plotting
\begin{verbatim}
      call iosys('open plot as scratch',0,0,0,' ')
      call iosys('create real wavefunction on plot',ntreg*2*n3d,0,0,' ') 
      do 30 tim=1,ntreg
         key(dim+1)=itoc(tim)
         len=length(key(dim+1))
         key(dim+1)='t'//key(dim+1)(1:len)
         len=length(key(dim+1))
         key(dim+1)='$v0('//key(dim+1)(1:len)//')'  
\end{verbatim}
     initialize the wavefunction at the first time or read in its value.
\begin{verbatim}
         t0=edge(1)
         time(1)=secnds(0.0)
         call cpsi0(ham1(eigv0(1)),ham2(eigv0(2)),ham3(eigv0(3)),
     1              ham1(eig0(1)),ham2(eig0(2)),ham3(eig0(3)),
     2              ham1(q(1)),ham2(q(2)),ham3(q(3)),
     3              y(psi0),energy,t0,n3d,nphy(1),
     4              dim,coord,tim,i0stat,card,prn(2))
         time(2)=secnds(0.0)
         delta(1)=time(2)-time(1)
\end{verbatim}
        Calculate the time dependent perturbation.  It consists of a  
        space and a time part.                                        
\begin{verbatim}
         call pert(y(v),vt,edge(1),ham1(q(1)),ham2(q(2)),ham3(q(3)),
     1             ham1(pq(1)),ham2(pq(2)),ham3(pq(3)),
     2             y(psi0),vtyp,scale,omega,width,shift,gamma,
     3             dim,n3d,nphy(1),spdim,coord,tim,prn(3))
         time(3)=secnds(0.0)
         delta(2)=time(3)-time(2)
\end{verbatim}
        Calculate the Hamiltonian times the initial state             
\begin{verbatim}
         call chpsi(ham1(h0(1)),ham2(h0(2)),ham3(h0(3)),
     1              y(driver),y(psi0),y(v),n3d,nphy(1),spdim,prn(4))
         time(4)=secnds(0.0)
         delta(3)=time(4)-time(3)
         delt=edge(tim+1)-edge(tim)
\end{verbatim}
        Propagate to the next time.  the approach depends on whether
        you use iterative or non iterative methods
\begin{verbatim}
         call prp3d(y(psi),y(driver),y(psi0),y(v),y(scr),
     1              ham1(eig0(1)),ham2(eig0(2)),ham3(eig0(3)),
     2              ham1(eigv0(1)),ham2(eigv0(2)),ham3(eigv0(3)),
     3              y(vexp),delt,spdim,n3d,nphy(1))
         call iosys('write real wavefunction to plot '//
     1              'without rewinding',2*n3d,y(psi),0,' ')
 30   continue   
      call iosys('rewind all on plot read-and-write',0,0,0,' ')
      call memory(-wdpsi,py,iidum,'psi',iidum)
      write(iout,1) (delta(i),i=1,3)
      write(iout,2) delta(4), delta(5)
      return
 1    format(/,1x,'time for right hand side   = ',e15.8,
     1       /,1x,'time for perturbation      = ',e15.8,
     2       /,1x,'time for driving term      = ',e15.8)
 2    format(/,1x,'time to propagate          = ',e15.8)
      end
\end{verbatim}
\end{document}
