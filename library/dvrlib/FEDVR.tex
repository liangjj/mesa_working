\documentstyle[preprint,aps]{revtex}

\def \<{\langle}

\def \>{\rangle}

\begin{document}

\title{Finite Element DVR}
\author{B. I. Schneider}

\address{ Physics Division, National Science Foundation, Arlington,
Virginia 22230 and Electron and Optical Physics Division, National
Institute of Standards and Technology, Gaithersburg, MD 20899}

\date{\today}

\maketitle

%

% ****** End of file template.aps ******

\section{DEFINITIONS}
The finite elements are constructed by breaking space up into pieces.  The
boundaries of element $i$ are defined as $R^{i}_{l}$ and $R^{i}_{r}$  Within each
element we define a set of $n_{i}$ DVR functions.  The number of functions in each
element is arbitrary except that we require that there be one DVR function at the
boundaries.  This is equivalent to requiring the Gauss quadrature to be of the
Lobatto variety; two fixed points.  The global basis is constructed from the 
set of all functions {\em internal } to all the elements plus the bridge functions.  While it completely arbitrary, I will define the bridge function 
associated with the $i^{th}$ region as,
\begin{equation}
  \psi^{i}_{q}(x)  = \big ( f^{i}_{N}(x) + f^{i+1}_{1}(x) \big ) 
\end{equation}
where $f^{i}$ are the last(first) DVR functions of the $i^{th}$ ( $(i+1)^{th}$ )
intervals.  So, by definition, the only connection between regions is via the
bridge functions.  The matrix elements are formed by integrating over $x$ and summing
over the regions.  The contributions to matrix elements of the internal DVR 
functions are confined to a single element.  A non-bridge function in region $i$
connects to bridge functions at the two ends of the interval.  Matrix elements between
bridge functions are connected to themselves and to adjacent bridge functions.  The
values of the matrix elements will be computed in the next section.  An important
point to note about the bridge functions is that they have a discontinuous

first derivative.  This requires care in computing any matrix elements involving
second derivatives, which do not exist.  Also, for too small a value of $n_{i}$ 
a discontinuous derivative may lead to poor behavior of the piecewise approximation.
Clearly, there is a relationship between the size of the interval and the basis set 
which needs to be explored for specific problems.  In practice I have observed 
that once one gets beyond 8 or 10 basis functions per interval, the issue of a 
discontinuous first derivative is no longer of practical consequence.  Importantly, 
a judicious use of a large number of intervals and a small basis set per interval, 
can be quite effective in substantially reducing the number of non-zero matrix 
elements in a given problem.  This has great import for computational techniques 
which rely heavily on sparse matrix manipulations.

\section{Matrix Elements}
In order to avoid the problem of the non-existence of the second derivative of the
bridge functions, one may formulate the problem variationally and then use Green's
theorem (integration by parts) to get a variationally weak condition which avoids
second derivatives.  Alternatively, one may simply use an interval by interval
Bloch operator approach in which the second derivative operaotr is replaced by
\begin{equation}
 L = \frac{d^2}{dx^{2}} - \delta(x-R_{r}) \frac{d}{dx} - \delta(x-R_{l}) \frac{d}{dx}
\end{equation}
We now compute the required integrals.  First the overlaps.  For non-bridge functions,
\begin{equation}
    O^{i}_{n,m} = \int_{R_{l}}^{R_{r}} dx f^{i}_{n}(x) f^{i}_{m}(x)
                = w^{i}_{n} \delta_{n,m}
\end{equation}
and for the bridge functions, 
\begin{equation}
        O^{i} = \int_{R_{l}}^{R_{r}} dx ( f^{i}_{N}(x) + f^{i+1}_{1}(x) )
                                        ( f^{i}_{N}(x) + f^{i+1}_{1}(x) )
              = O^{i}_{N,N} + O^{i+1}_{1,1}
              = w^{i}_{N} + w^{i+1}_{1}
\end{equation}
So, all the overlaps are diagonal, as expected.
Now, turn to the $L$ matrix elements. The non-bridge matrix elements are,
\begin{equation}
    L^{i}_{n,m} = \int_{R_{l}}^{R_{r}} dx f^{i}_{n}(x)L f^{i}_{m}(x)
                = w^{i}_{n} f^{ \prime \prime i }_{m}(x^{i}_{n})
\end{equation}
The matrix elements between a non-bridge and bridge function are,
\begin{eqnarray}
    L^{i,i}_{n,m} &=& \int_{R_{l}}^{R_{r}} dx f^{i}_{n}(x) L 
                      ( f^{i}_{N}(x) + f^{i+1}_{1}(x) ) \\ \nonumber
                &=& L^{i}_{n,N} \\ \nonumber
    L^{i,i-1}_{n,m} &=& \int_{R_{l}}^{R_{r}} dx f^{i}_{n}(x) L 
                      ( f^{i}_{1}(x) + f^{i-1}_{N}(x) ) \\ \nonumber
                &=& L^{i}_{n,1} 
\end{eqnarray}
Finally, the bridge-bridge matrix elements are,
\begin{eqnarray}
    L^{i,i} &=& \int_{R_{l}}^{R_{r}} dx ( f^{i}_{N}(x) + f^{i+1}_{1}(x) ) L 
                      ( f^{i}_{N}(x) + f^{i+1}_{1}(x) ) \\ \nonumber
            &=& L^{i}_{N,N} + L^{i+1}_{1,1} \\ \nonumber
    L^{i-1,i} &=& \int_{R_{l}}^{R_{r}} dx ( f^{i-1}_{N}(x) + f^{i}_{1}(x) ) L 
                      ( f^{i}_{N}(x) + f^{i+1}_{1}(x) ) \\ \nonumber
            &=& L^{i}_{1,N} \\ \nonumber
    L^{i+1,i} &=& \int_{R_{l}}^{R_{r}} dx ( f^{i+1}_{N}(x) + f^{i+2}_{1}(x) )
                    L ( f^{i}_{N}(x) + f^{i+1}_{1}(x) ) \\ \nonumber
            &=& L^{i+1}_{N,1} 
\end{eqnarray}

\section{PRACTICALITIES}
Let us work out the basic integral.  We transform to the standard (-1,1) for
finite intervals.
\begin{eqnarray}
   L^{i}_{n,m} &=& \int_{R_{l}}^{R_{r}} dx f^{i}_{n}(x) L f^{i}_{m}(x) 
                                   \\ \nonumber
                &=& \frac{2}{(R^{i}_{r} - R^{i}_{l})} \int_{-1}^{1} dx
                   f^{i}_{n}(x) \frac{d^2 f^{i}_{m}(x)}{dx^2}
                  - f^{i}_{n}(1) f^{\prime i}_{m}(x) \mid_{x=1}  
                  + f^{i}_{n}(-1) f^{\prime i}_{m}(x) \mid_{x=-1}  
\end{eqnarray}
Clearly the surface terms vanish unless one of the functions is at the ends
of the interval

\end{document}

