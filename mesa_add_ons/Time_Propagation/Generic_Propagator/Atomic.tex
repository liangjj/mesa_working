\documentclass[preprint,showpacs,preprintnumbers,amsmath,amssymb]{revtex4}

\def \<{\langle}

\def \>{\rangle}

\begin{document}

\title{Atomic Calculations using Finite Element DVR}
\author{B. I. Schneider}

\address{ Physics Division, National Science Foundation, Arlington,
Virginia 22230 and Electron and Optical Physics Division, National
Institute of Standards and Technology, Gaithersburg, MD 20899}

\date{\today}

\maketitle

%

% ****** End of file template.aps ******

\section{Introduction}
The purpose of these notes is to outline a plan which uses the finite 
element discrete variable method(FEDVR) to perform atomic structure and 
scattering calculations.  While any weight function is acceptable, here 
we restrict ourselves to the Legendre weight function, w(r)=1 and the 
points and weights of the Gauss-Legendre-Lobatto points mapped onto the 
domain of each element.  Thus in each element, we have a set of DVR 
functions of the form,
\begin{equation}
  \chi^{i}_{qlm}(r,\Omega)  = Y_{lm}(\Omega) \psi^{i}_{q}(r) 
\end{equation}
where the set of points in interval $i$ is required by the Lobatto rule 
to include the two endpoints.  If the boundary conditions require it, 
either the first, last or both functions in a sub-interval may be absent 
from the basis.  This is often necessary to satisfy the condition of 
regularity at the origin and decay of the bound-state wavefunctions 
at large radial distances.  In order to connect up two adjacent regions, 
the last DVR function in region $i$ and the first DVR function in 
region $i+1$ are linearly combined to get a bridge function,
\begin{equation}
  \hat \psi^{i}_{n-1}(r)  = \psi^{i}_{n}(r) + \psi^{i+1}_{1}(r).  
\end{equation}
Note that this function has a discontinuous derivative at the boundary 
which needs careful treatment.  The collection of all of the bridge and 
non-bridge functions form the basis.

\section{Matrix Elements}
\subsection{one electron integrals}
Since all of the integrals may be computed by integrating over the 
sub-domains and then summing the results, with the exception of integrals 
involving the derivatives of the bridge functions, all are straightforward.
Since the kinetic energy operator contains a second derivative, it is 
necessary to ``soften'' the problem variationally by using Green's theorem 
(integration by parts) to get what is called, a variationally weak 
condition which avoids the non-existence of the second derivative at the 
boundary.  This is functionally equivalent to modifying the second 
derivative by adding a Bloch operator in each interval.  Thus we replace the
radial kinetic energy operator in each interval by
\begin{equation}
 L = -\frac{1}{2} \frac{d^2}{dr^{2}}  
                    + \frac{1}{2} \delta(r-r_{right}) \frac{d}{dr} 
                    - \frac{1}{2} \delta(r-r_{left}) \frac{d}{dr} 
\end{equation}
We now compute the required integrals.  First the overlaps.  For non-bridge 
functions,
\begin{equation}
    O^{i,j}_{p,q} = \int dr \psi^{i}_{p}(r) \psi^{i}_{q}(r)
                = w^{i}_{p} \delta_{i,j} \delta_{p,q}
\end{equation}
and for the bridge functions, 
\begin{equation}
        O^{i,i}_{n-1,n-1} = \int dr ( \psi^{i}_{n}(r) + \psi^{i+1}_{1}(r) )
                        ( \psi^{i}_{n}(r) + \psi^{i+1}_{1}(r) )
              = O^{i,i}_{n,n} + O^{i+1,i+1}_{1,1}
              = w^{i}_{n} + w^{i+1}_{1}
\end{equation}
So, all the overlaps are diagonal, as expected.
Matrix elements involving any local potential, including the centrifugal
potential, are also diagonal.  Thus,
\begin{equation}
    U^{i,j}_{p,q} = \int dr \psi^{i}_{p}(r) U(r) \psi^{i}_{q}(r)
                = w^{i}_{p} U(r^{i}_{p}) \delta_{i,j} \delta_{p,q}
\end{equation}
and for the bridge functions, 
\begin{equation}
        U^{i,i}_{n-1,n-1} = \int dr ( \psi^{i}_{n}(r) + \psi^{i+1}_{1}(r) )
                   U(r)  ( \psi^{i}_{n}(r) + \psi^{i+1}_{1}(r) )
              = U^{i,i}_{n,n} + U^{i+1,i+1}_{1,1}
              = ( w^{i}_{n} + w^{i+1}_{1} ) U(r^{i}_{n})
\end{equation}
Note that the normalization for the bridge functions removes the double
weighting in the final normalized matrix elements.
Now, turn to the $L$ matrix elements. The non-bridge matrix elements are,
\begin{equation}
    L^{i,i}_{p,q} = \int dr \psi^{i}_{p}(r) L \psi^{i}_{q}(r)
                = w^{i}_{p} \psi^{ \prime \prime i }_{q}(r^{i}_{p})
\end{equation}
The matrix elements between a non-bridge and bridge function are,
\begin{eqnarray}
    L^{i,i}_{p,n-1} &=& \int dr \psi^{i}_{p}(r) L 
                      ( \psi^{i}_{n}(r) + \psi^{i+1}_{1}(r) ) \\ \nonumber
                &=& L^{i,i}_{p,n} \\ \nonumber
    L^{i,i-1}_{p,m-1} &=& \int dx \psi^{i}_{p}(r) L 
                      ( \psi^{i}_{1}(r) + \psi^{i-1}_{m}(r) ) \\ \nonumber
                &=& L^{i,i}_{p,1} 
\end{eqnarray}
Finally, the bridge-bridge matrix elements are,
\begin{eqnarray}
    L^{i,i}_{n-1,n-1} &=& \int dr ( \psi^{i}_{n}(r) + \psi^{i+1}_{1}(r) ) L 
                      ( \psi^{i}_{n}(r) + \psi^{i+1}_{1}(r) ) \\ \nonumber
            &=& L^{i,i}_{n,n} + L^{i+1,i+1}_{1,1} \\ \nonumber
    L^{i,i-1}_{n-1,m-1} &=& \int dr ( \psi^{i}_{n}(r) + \psi^{i+1}_{1}(r) )
                    L ( \psi^{i-1}_{m}(r) + \psi^{i}_{1}(r) ) \\ \nonumber
            &=& L^{i,i}_{n,1} 
\end{eqnarray}
Thus, we see that there are never contributions to the matrix elements
coming from cross interval integrations.  Global matrix elements are
simply formed from linear combinations of matrix elements diagonal in the
element indices.  The entire calculation is simply reduced to computing,
\begin{eqnarray}
    L^{i,i}_{p,q} &=& \int dr \psi^{i}_{p}(r) L ( \psi^{i}_{q}(r) 
                                        \\ \nonumber
       &=& \int_{r_{left}}^{r_{right}} dr \psi^{i}_{p}(r) L \psi^{i}_{q}(r)
\end{eqnarray}
where we allow the indices $p$ and $q$ to be $1$ and $n$.
If we transform to the standard interval (-1,1), we get
\begin{eqnarray}
   L^{i,i}_{p,q} &=& - \frac{1}{r_{right} - r_{left} }
     \int_{-1}^{1} dr f\psi^{i}_{p}(x) \frac{ d^2 \psi^{i}_{q}(x)} {dx^2} 
         + \frac{1}{2} \psi^{i}_{p}(1) \psi^{\prime i}_{q}(x) \mid_{x=1}  
         - \frac{1}{2} \psi^{i}_{p}(-1) \psi^{\prime i}_{q}(x) \mid_{x=-1}  
\end{eqnarray}
Clearly the surface terms vanish unless $p$ is either $1$ or $n$. 
For the remaining discussion, we assume we have re-normalized
the FEDVR functions so that their INTEGRAL is unity.
\subsection{two electron integrals}
If we define atomic densities as the product of two of the orbitals, the 
two electron integrals may simply be written as,
\begin{equation}
    V_{ij,kl} = \int d{\bf r_1} d{\bf r_2}  \rho_{ik}({\bf r_1}) 
                   \frac{1}{{\bf r_1} - {\bf r_2} } \rho_{jl}({\bf r_2 }) 
\end{equation}
Each of the densities is given as,
\begin{equation}
 \rho_{ij}({\bf r})  = \sum_{LM} d^{LM}_{l_{i}m_{i}l_{j}m_{j}} 
                                 Y_{LM}(\Omega) 
                       \gamma_{ij}(r) 
\end{equation}
where the coupling of the spherical harmonics of the individual orbitals
follows the usual Clebsch-Gordan rules.  We may also expand the interaction
potential in spherical harmonics,
\begin{equation}
    \frac{1}{{\bf r_1} - {\bf r_2} } = 4 \pi \sum_{lm} \frac{1}{(2l+1)} 
                                    Y_{lm}(\Omega_{1}) Y_{lm}(\Omega_{2}) 
                                      \frac{r_{<}^{l}}{r_{>}^{l+1}}
\end{equation}
The two electron integral may then be reduced to,
\begin{eqnarray}
    V_{ij,kl} = 4 \pi \sum_{LM} \frac{1}{(2L+1)} 
                d^{LM}_{l_{i}m_{i}l_{k}m_{k}} d^{LM}_{l_{j}m_{j}l_{n}m_{n}} 
                  \int r^{2}_{1} dr_1 r^{2}_{2} dr_2
                  \gamma_{ik}(r_1) \frac{r_{<}^{L}}{r_{>}^{L+1}}
                   \gamma_{jn}(r_2)      \\ \nonumber
              = 4 \pi \sum_{LM} \frac{1}{(2L+1)} 
                d^{LM}_{l_{i}m_{i}l_{k}m_{k}} d^{LM}_{l_{j}m_{j}l_{n}m_{n}} 
                V^{L}_{ij,kl}
\end{eqnarray}
The essential difficulty with this integral is the derivative discontinuity
of the radial matrix element.  A naive application of the quadrature rule
relating the DVR functions to simple sums over the values of functions at
the quadrature points, yields poor accuracy.  However, it is possible to 
avoid this and to retain high accuracy by using Poisson's equation.  
We define
\begin{equation}
    U^{L}_{ik}(r) = \int r_{1}^{2} dr_{1} 
                     \gamma_{ik}(r_1) \frac{r_{<}^{L}}{r_{>}^{L+1}}      
\end{equation}
which is equivalent to the radial differential equation,
\begin{equation}
 \frac{1}{r^2} \frac{d}{dr} r^2 \frac {d}{dr} 
                 U^{L}_{ik}(r) 
              - \frac{L(L+1)}{r^2} U^{L}_{ik}(r) 
               = - (2L+1) \gamma_{ik}(r)      
\end{equation}
In writing this I assume that the orbitals are defined to behave as 
$r^{l+1}$ at the origin, so that the radial density is actually the product
of two orbitals divided by $r^2$  This has the effect of cancelling a 
factor of $\frac{1}{r^2}$ from both sides of the above equation. 
\begin{equation}
 \frac{d}{dr} r^2 \frac {d}{dr} U^{L}_{ik}(r) - L(L+1) U^{L}_{ik}(r) 
              = - (2L+1) \psi^{t}_{i}(r) \psi^{v}_{k}(r)
\end{equation}
We now define,
\begin{equation}
 U^{L}_{ik}(r)= \frac{ V^{L}_{ik}(r) }{r}
\end{equation}
to get,
\begin{equation}
 \frac{d^2}{dr^2} V^{L}_{ik}(r) - \frac{L(L+1)}{r^2} V^{L}_{ik}(r) 
              = - (2L+1) \frac {\psi^{t}_{i}(r) \psi^{v}_{k}(r) }{r}
\end{equation}
To solve this differential equation requires two boundary conditions.  
These can be discovered by looking at the definition of $U^{L}$,
\begin{equation}
  U^{L}_{ik}(r) = \frac{1}{r^{L+1}} \int_{0}^{r} dr_{1} 
                   \psi^{t}_{i}(r_{1}) \psi^{v}_{k}(r_{1}) r_{1}^{L}
                  + r^{L} \int_{r}^{r_{N}} dr_{1} 
         \frac { \psi^{t}_{i}(r_{1}) \psi^{v}_{k}(r_{1}) }{r^{L+1}}
\end{equation}
At $r=0$, $U^{L}$ behaves like $r^{L}$ and the multiplicative $r$ factor
which converts to $V^{L}$, gives a zero boundary condition at the origin
for all $L$.  At the last point, $U^{L}$ behaves like $\frac{1}{r^{L+1}}$ 
and the multiplicative $r$ factor gives,
\begin{equation}
 V^{L}_{ik}(r_{N}) = \frac{1}{r_{N}^{L}} \int_{0}^{\infty} dr 
                   \psi^{t}_{i}(r) \psi^{v}_{k}(r) r^{L}
                   = \frac{r^{L}_{i}}{r^{L}_{N}} \delta_{i,k} 
\end{equation}
This extremely simple result is a consequence of the use of the DVR basis.
So, now we may solve eq(21).  The simplest formal procedure is to solve
the differential equation for a solution which is zero at both boundaries
and to then add a solution of the homogeneous equation satisfying the
inhomogeneous boundary condition at the last point.  
\begin{equation}
 V^{L}_{ik}(r) = \sum_{j \neq N } c^{ik}_{j} \psi_{j}(r) 
                  +  r^{L+1} \frac {r^{L}_{i}}{r^{2L+1}_{N}} \delta_{i,k}
\end{equation}
Note that the last DVR function is excluded from the sum to ensure a 
zero value at the last point.
Inserting the expression about into equation(21) and projecting onto the 
basis yields,
\begin{equation}
 \sum_{t} T_{it} c^{jl}_{t}  = - (2L+1) \delta_{i,j} 
                            \delta_{j,l} \frac{1}{\sqrt{w_{i}} r_{i}}
\end{equation}
The matrix $T$ is defined as,
\begin{equation}
 T_{ij} = \langle \psi_{i} \mid \frac{d^2}{dr^2} - \frac{L(L+1)}{r^2}
           \mid \psi_{j} \rangle
\end{equation}
and the integration must be performed using the Bloch operator for the
second derivative.  The formal solution to this equation is,
\begin{equation}
 c^{jl}_{i}  = - (2L+1) T^{-1}_{ij} \delta_{j,l} 
                        \frac{1}{\sqrt{w_{j}} r_{j}}
\end{equation}
Using this yields the final expression for $U^{L}_{jl}$
\begin{equation}
 U^{L}_{jl}(r) = -(2L+1) \frac{ \delta_{j,l} }{r r_{j} \sqrt{w_{j}} }
               \sum_{i} T^{-1}_{ij} \psi_{i}(r)  
               + \delta_{j,l} \frac{ r^{L} r^{L}_{j} }{r^{2L+1}_{N}}
\end{equation}
and therefore,
\begin{equation}
  V^{L}_{ij,kl} = \delta_{i,k} \delta_{j,l} \big [ 
                    -(2L+1) \frac{ T^{-1}_{ij} }{ r_{i} \sqrt{w_{i}}
                                                  r_{j} \sqrt{w_{j}} }
                    + \frac{ r^{L}_{i} r^{L}_{j } } { r^{2L+1}_{N} }
                    \big ]
\end{equation}
\section{Hamiltonian}
The next step in the process is the construction of the Hamiltonian matrix
from the one-and-two-electron integrals.  Since there is nothing physical about
the FEDVR basis, a first step, would be to set up some model, single particle
Hamiltonian, preferably with a local but realistic interaction, and diagonalize
it to get a single set of orbitals.  It would be possible to use the Fock
Hamiltonian but that would require using the two-electron integrals.  Not
impossible, but perhaps more than is needed.  
\par
So, initially, we set up a one-body problem using the FEDVR.  Let us consider
the structure of the one-body Hamiltonian matrix for two finite elements with
3 basis functions in the first and 2 in the second element.
 \[ \left ( \begin{array}{lllll}
(H_{1,1} - E) & H_{1,2} & 0 & 0 & 0 \\
H_{2,1} & (H_{2,2} - E ) & H_{2,3} & 0 & 0 \\
0 & H_{3,2} & ( H_{3,3} - E ) & H_{3,4} & 0 \\
0 & 0 & H_{4,3} & ( H_{4,4} - E ) & H_{4,5} \\
0 & 0 & 0 & H_{5,4} & ( H_{5,5} - E )
\end{array} \right ) \]
If we were to consider the first element as an ``internal'' region, where
exchange and correlation effects were important and the second element as an
``external'' region, dominated by long-range forces, the connection is
provided by the ``bridge'' function which straddles the two regions.  A 
reasonable approach would be ignore region two and just diagonalize the
leading (3x3) submatrix to get a set of ``internal'' orbitals.  These orbitals
would be used to build the many-electron spin-eigenfunctions required for
the target states and the pseudostates.
\end{document}






